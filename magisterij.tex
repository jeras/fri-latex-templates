%*****************************************************************
%   Vzorec za pisanje diplomskega dela,
%   ki vsebuje navodila za izdelavo diplomskega dela
%
%   UNIVERZA V LJUBLJANI
%   Fakulteta za računalništvo in informatiko
%
%   Pripravila: Peter.Peer@fri.uni-lj.si
%               Franc.Solina@fri.uni-lj.si
%*****************************************************************

\documentclass[12pt,a4paper,openany]{book}

%Uporabljeni paketi
\usepackage[utf8]{inputenc}
\usepackage{cmap}
\usepackage{type1ec}
\usepackage[T1]{fontenc}
\usepackage{fancyhdr}
\usepackage{graphicx,epsfig}
\usepackage[slovene]{babel}
\usepackage{cite}
\usepackage{enumitem}
\usepackage{amsmath}

\usepackage[pdftex,colorlinks,citecolor=black,filecolor=black,linkcolor=black,urlcolor=black,pagebackref]{hyperref}
\usepackage{tikz}

%Velikost strani - dvostransko
\oddsidemargin 1.4cm
\evensidemargin 0.35cm
\textwidth 14cm
\topmargin 0.26cm
\headheight 0.6cm
\headsep 1.5cm
\textheight 20cm

%Nastavitev glave in repa strani
\pagestyle{fancy}
\fancyhead{}
\renewcommand{\chaptermark}[1]{\markboth{\textsf{Poglavje \thechapter:\ #1}}{}}
\renewcommand{\sectionmark}[1]{\markright{\textsf{\thesection\  #1}}{}}
\fancyhead[RE]{\leftmark}
\fancyhead[LO]{\rightmark}
\fancyhead[LE,RO]{\thepage}
\fancyfoot{}
\renewcommand{\headrulewidth}{0.0pt}
\renewcommand{\footrulewidth}{0.0pt}

\newcommand{\gnuplot}{\textbf{gnuplot}}
\newcommand{\pgfname}{\textsc{pgf}}
\newcommand{\tikzname}{Ti\emph{k}Z}

\input{cc}

%********************************************
% kratice, simboli
\newcommand{\abbrlabel}[1]{\makebox[3cm][l]{\textbf{#1}\ \dotfill}}
\newenvironment{abbreviations}{\begin{list}{}{\renewcommand{\makelabel}{\abbrlabel}}}{\end{list}}

%********************************************

\begin{document}

% stran 1 med uvodnimi listi
\thispagestyle{empty} 

\begin{center}
{\large 
UNIVERZA V LJUBLJANI\\
FAKULTETA ZA RAČUNALNIŠTVO IN INFORMATIKO\\
}

\vspace{3cm}
{\LARGE Iztok Jeras}\\

\vspace{2cm}
\textsc{\textbf{\LARGE 
Predslike 2D celičnih avtomatov
}}

\vspace{2cm}
{ MAGISTERSKO DELO}\\
{ NA UNIVERZITETNEM ŠTUDIJU}\\

\vspace{2cm} 
{\Large Mentor: prof. dr. Branko Šter}

\vfill
{\Large Ljubljana, 2016}
\end{center}

\newpage

\ \thispagestyle{empty}

\newpage

%********************************************

% stran 2 med uvodnimi listi
\thispagestyle{empty}

\vspace*{5cm}
{\small \noindent
To magistrsko delo je ponujeno pod licenco \textit{Creative Commons Priznanje avtorstva-Deljenje pod enakimi pogoji 2.5 Slovenija}
ali (po želji) novejšo različico.
To pomeni, da se tako besedilo, slike, grafi in druge sestavine dela kot tudi rezultati diplomskega dela lahko prosto distribuirajo,
reproducirajo, uporabljajo, dajejo v najem, priobčujejo javnosti in predelujejo, pod pogojem, da se jasno in vidno navede avtorja in naslov tega
dela in da se v primeru spremembe, preoblikovanja ali uporabe tega dela v svojem delu, lahko distribuira predelava le pod
licenco, ki je enaka tej.
Podrobnosti licence so dostopne na spletni strani \url{http://creativecommons.si/} ali na Inštitutu za
intelektualno lastnino, Streliška 1, 1000 Ljubljana.

\begin{center}% 0.66 / 0.89 = 0.741573033707865
  \CcImageCc{0.741573033707865}\hspace*{1ex}\CcGroupBySa{1}{1ex}
\end{center}
}

\vspace*{1.5cm}
{\small \noindent
Izvorna koda diplomskega dela, njenih rezultatov in v ta namen razvite programske opreme je ponujena pod GNU General Public License,
različica 3 ali (po želji) novejšo različico. To pomeni, da se lahko prosto uporablja, distribuira in/ali predeluje pod njenimi pogoji.
Podrobnosti licence so dostopne na spletni strani \url{http://www.gnu.org/licenses/}.
}

\begin{center} 
\ \\ \vfill
{\em
Besedilo je oblikovano z urejevalnikom besedil \LaTeX. \\ Slike so izdelane s pomočjo jezika \pgfname/\tikzname. \\ Grafi so narisani
s pomočjo programa \gnuplot.}
\end{center}

\newpage

\ \thispagestyle{empty}

\newpage

%********************************************

% stran 3 med uvodnimi listi
\thispagestyle{empty}

Namesto te strani {\bf vstavite} original izdane teme diplomskega dela s podpisom mentorja in dekana ter \v zigom fakultete, ki ga diplomant
dvigne v študent\-skem referatu,  preden odda izdelek v vezavo!

\newpage

%********************************************

% stran 4 med uvodnimi listi je prazna 
\ \thispagestyle{empty}

\newpage

%********************************************

% stran 5 med uvodnimi listi

\thispagestyle{empty}

\vspace{1cm}
\begin{center} 
{\Large \textbf{IZJAVA O AVTORSTVU}}
\end{center}

\begin{center} 
{\Large diplomskega dela}
\end{center}

\vspace{1cm}
Spodaj podpisani \hspace{0.5cm} Iztok Jeras,

\vspace{0.5cm}
z vpisno številko \hspace{0.5cm} 63030393,

\vspace{1cm}
sem avtor diplomskega dela z naslovom:
   
\vspace{0.5cm}
Predslike 2D celičnih avtomatov

\vspace{1.5cm}
S svojim podpisom zagotavljam, da:
\begin{itemize}
	\item sem diplomsko delo izdelal samostojno pod mentorstvom 
	
	prof. dr. Branko Šter
	
	in somentorstvom 
	
	prof. [doc.] dr. Ime Priimek
	
	\item so elektronska oblika diplomskega dela, naslov (slov., angl.), povzetek (slov., angl.) ter ključne besede (slov., angl.) identični s tiskano obliko diplomskega dela
	\item soglašam z javno objavo elektronske oblike diplomskega dela v zbirki ''Dela FRI''.
\end{itemize}

\vspace{1cm}
V Ljubljani, dne xx.xx.2016 \hspace{1cm} Podpis avtorja/-ice:

\newpage 

%********************************************

% stran 6 med uvodnimi listi je prazna pri dvostranskem tiskanju
\ \thispagestyle{empty}

\newpage

%********************************************

% stran 7 med uvodnimi listi

\chapter*{Zahvala}

\thispagestyle{empty}

Na tem mestu se diplomant zahvali vsem, ki so kakorkoli pripomogli k uspešni izvedbi diplomskega dela.


\newpage

%********************************************

% stran 8 med uvodnimi listi je prazna pri dvostranskem tiskanju
\ \thispagestyle{empty}

\newpage

%********************************************

% stran 9 med uvodnimi listi
\thispagestyle{empty}

$\;$ 

\vspace{5cm}
\hfill {\Large \em Morebitno posvetilo}
\thispagestyle{empty}

\newpage

%********************************************

% stran 10 med uvodnimi listi je prazna pri dvostranskem tiskanju

\ \thispagestyle{empty}

\newpage

%********************************************

\renewcommand\thepage{} 
\tableofcontents 
\renewcommand\thepage{\arabic{page}}

\thispagestyle{empty}


%********************************************

\chapter*{Seznam uporabljenih kratic in simbolov}

\thispagestyle{empty}

% simboli

\begin{abbreviations}
\item[1D] eno dimenzionalen
\item[2D] dvo dimenzionalen
\item[CA] celični avtomat
\item[GoL] Game of Life (igra življenja)
\item[GoE] Garden of Eden (stanje brez predslik)
\item[trid] okolica CA sestevljena iz treh celic
\item[quad] okolica CA sestevljena iz štirih celic \(M_x=M_y=2\)
\end{abbreviations}

%\cleardoublepage

\clearpage{\pagestyle{empty}\cleardoublepage}

%********************************************
%zacno se glavni listi, ki so numerirani z arabskimi stevilkami

\setcounter{page}{1}
\pagenumbering{arabic}

\chapter*{Povzetek}

\addcontentsline{toc}{chapter}{Povzetek}

Medtem ko je računanje predslik 1D celičnih avtomatov dobro raziskan in
podrobno dokumentiran problem, je to področje pri 2D celičnih avtomatih manj
raziskano. To magistersko delo poizkuša aplicirati metode razvite za 1D avtomate
na 2D problem. Prikazan je algoritem, ki omogočajo štetje in izpis predslik.
Razvit je bil s pomočjo grafičnega modela, ki omogoča uporabo teorije grafov pri izračunih in dokazih.

\vspace{1.3cm}
\noindent
{\large \bf Ključne besede:}

\vspace{0.5cm}
\noindent
celični avtomati, predslike, procesna zahtevnost, reverzibilnost, trid, quad


\chapter*{Abstract}

\addcontentsline{toc}{chapter}{Abstract}

While computing preimages of 1D cellular automata is a well researched and
documented problem, for 2D cellular automata there is less research available.
This masters thesis attempts to apply methotds developed for 1D automata to the
2D problem. An algorithm is shown, which can count and list preimages.
It was developed with the help of a graphical representation, which enables using graph theory for computation and proofs.

\vspace{1.3cm}
\noindent
{\large \bf Key words:}

\vspace{0.5cm}
\noindent
cellular automata, preimages, computational complexity, reversibility, Garden of Eden, Game of Life, trid, quad


%********************************************

\chapter{Uvod}

\section{Celični avtomat kakor model vesolja}

Ker lahko vsak univerzalen sistem modelira vsak drugi univerzalen sistem, predpostavimo,
da lahko z univerzalnimi CA modeliramo vesolje. Samo modeliranje vesolja
je še izven našega dosega, poizkuša pa se vsaj doseči povezave z teoretično fiziko.
S stališča informacijske teorije in termodinamike je predvsem zanimiv model gravitacije
kakor entropijske sile (Entropic gravity \cite{Verlinde2010}), ki predpostavlja, da je
3D vesolje projekcija procesov, ki se odvijajo na 2D ploskvi. Podobno CA omogočajo
opazovanje abstraktnega kopiranja informacij (replikacija) in evolucijo \cite{Salzberg2004}.

\section{Informacijska dinamika}

Informacijsko dinamiko CA se najpogosteje opisuje samo kakor reverzibilno ali ireverzibilno,
obstaja tudi nekaj člankov, ki opazujejo entropijo sistema.
Pogosto je tudi opazovanje dinamike delcev pri Game of Life ali elementarnem pravilu 110.
Ne obstaja pa še splošna teorija dinamike informacij v CA.
V svojem članku \cite{JerasDobnikar2007} in prispevkinh na konferencah \cite{DBLP:conf/iccS/JerasD06, DBLP:conf/automata/Jeras08},
sem grafično upodobil predslike trenutnega stanja za 1D problem.
Iz upodobitve je videti, da se ponekod izgubi več informacije kakor drugod,
kar kaže na možnost izpeljave kvalitativne in kvantitativne teorije dinamike informacij;
žal se ta možnost še ni udejanila. Podobno je možno grafično upodobiti predslike 2D CA,
ter iz grafov sklepati o izgubi informacij v 2D CA.

\section{Problem predslik 2D CA}

Najbolje teoretično raziskan 2D CA je GoL (Game of Life ali slovensko igra živjenja).
Ogromno truda je bilo vloženega v raziskovanje delcev in njihove dinamike. S pomočjo
osnovnih gradnikov, je mogoče skonstruirati kompleksnejše sisteme, med katerimi so
najzanimivejši turingov stroj /cite{} in univerzalni konstruktor /cite{}.

Delci so dejansko atraktorji v razvoju CA na končni običajno periodični mreži (thorus).
Algoritmi za iskanje predslik so uporabljeni za določitev atraktorjevega korita /cite{}.
V ireverzibilnem celičnem avtomatu se pojavljajo stanja brez predslik imenovana GoE
(Garden of Eden ali rajski vrt). Pomensko so GoE nasprotje delcev, saj se nahajajo kar
najdlje od atraktorja na robu korita. GoE stanja prav tako kakor delci privlačijo raziskovalce /cite{},
čeprav v manši meri kakor delci.

Največ raziskav s področja predslik GoL je bilo opravljenih ravno s ciljem iskanja GoE stanj /cite{}.
S stališča algoritma za štetje predlik je GoE stanje tako, ki ima 0 preslik.
Algoritem za štetje predslik je možno pretvoriti v algoritem za preverjanje ali je stanje GoE,
tako da se opracije nad celimi števili pretvori v logične opreracije nad Boolovimi staji.

Raziskave algoritmov sem se lotil s predpostavko, da je možno opraviti štetje s zahtevnostjo,
ki je linearno odvisna od velikosti problema \( O(n) \)  (n je število opazovanih celic). Čeprav je to možno pri 1D problemu,
se izkaže da 2D problem ni tako preprost. Predstavljeni so primeri iz katerih je razvidno,
da algoritem z linearno zahtevnostjo ne more pravilno opisati vseh situacij.
Zahtevnost opisanega algotitma sicer raste s kvadratom eksponenta eksponentno z eno
od dimenzij CA mreže  \( O(2^2n) \), ja pa možno da obstaja algiritem z nižjo kompleksnostjo.

Opisan algoritem je primerjan z ostalimi doslej znanimi algoritmi. Čeprav s stališča procesne
zahtevnosti ne prinaša želenega napredka, pa to da temelji na pregledni grafični upodobitvi
daje upanje, da bojo razne optimizacije razvidne bodočim raziskovalcem.

Paulina Léon in Genaro Martínez \cite{PaulinaGenaro2016} poizkušata aplicirati
De Bruijn-ove diagrame na 2D celične avtomate, doslej je bilo to orodje uporabljeno le na 1D problemih.
Točneje, opazujeta dva celična avtomata:'Game of Life' in 'Diffusion rule',
s poudarkom na opazovanju stabilnih delcev. De Bruijn-ovi grafi so tudi osnova mojih raziskav,
so pa drugače grafično upodobljeni, tako da se jih lahko poveže v opis celotnega celičnega sistema,
in niso omejeni na opis predstanj ene same celice.

Razni avtorji \cite{Hartman2013} iščejo vzorce tipa 'Garden of Eden' v celičnem avtomatu 'Game of Life'.
Zanimiv je pristop s teorijo končnih avtomatov in regularnih jezikov, ki je v
osnovi namenjen eno dimenzionalnim sistemom. Jean Hardouin-Duparc ga je razširil
tako, da je celice iz vrstice 2D polja združil v simbole regularnega jezika, zaporedje več
vrstic pa predstavlja besedo. Originalni članek je v francoščini, zato še iščem članek,
kjer bi pristop opisal v angleščini. Podoben pristop s končnimi avtomati nameravam uporabiti tudi sam.

Doslej sem že razvil napredne algoritme za izračun predslik 1D sistema.
Skozi zgodovino so taki algoritmi napredovali, tako da je padala njihova
procesna zahtevnost in opisna/implementacijska zahtevnost.
\begin{enumerate}[noitemsep,nolistsep]
 \item 'brute force' algoritmi \( O(c^n) \)
 \item improvizirani algoritmi
 \item zasnove matematičnega modela
 \item optimalni algoritmi \( O(n \log(n)) \) ali celo \( O(n) \)
\end{enumerate}

Iskanje slik 2D sistema je trenutno nekje med improvizacijo in matematičnim modelom.
Z magistrsko nalogo bi rad razvil algoritme, ki se nagibajo k optimalnosti.

Magistrsko delo bo obsegalo matematičen model, ki bo predvidoma temeljil na matričnih operacijah,
kjer matrike predstavljajo grafe in končne avtomate.
Za lažje razumevanje bo problem tudi grafično predstavljen.
Algoritem bo implementiran kot računalniški program,
ki bo omogočal tudi izris grafične predstavitve problema.

Primerjava s sorodnimi deli bo s stališča procesne zahtevnosi algoritma in glede na to,
katere znane probleme bo algoritem sposoben rešiti. Nekaj takih problemov, urejenih glede na zahtevnost, je:
\begin{enumerate}[noitemsep,nolistsep]
\item določitev, ali obstajajo predslike za dano trenutno stanje sistema
\item štetje predslik
\item naštevanje konfiguracij predslik
\item jezik vseh stanj brez predslik
\item vprašanje reverzibilnosti sistema
\end{enumerate}

Rešitev problema določitve obstoja predslik si že predstavljam. 
Predvidevam, da bom uspel rešiti še problem preštevanja predslik,
in ker je to manjši korak, tudi njihovo naštevanje.

Preostalih problemov se tokrat ne bom loteval.
Problem jezika stanj brez predslik bi potreboval teorijo 2D formalnega jezika, ki še ne obstaja.
Poleg tega je povezan s problemom reverzibilnosti, ki je na spološno dokazano nerešljiv \cite{Kari1989}.

\chapter{Definicija 2D celičnega avtomata}

Predstavljena definicija 2D avtomata je enostavna in manj formalna
v primerjavi z definicijo 1D avtomata v podobnih člankih.
Bolj podrobna definicija ni potrebna, saj se opisani problemi za 2D avtomate,
še ne povezujejo z drugimi vejami matematike toliko kakor 1D avtomati.

V grafičnih upodobitvaj je uporabljena izometrična projekcija,
ker je za potrebe analize osnovni 2D mreži dodana tretja dimenzija,
ki opisuje prostor predslik.

Osnovni element 2D celičnega avtomata je celica kakor del polja celic.
Vsaka celica ima diskretno vrednost \(c\) iz nabora stanj celice \(S\).
Stanja so običajno kar oštevilčena.

\[ c \in S
   \quad \textrm{and} \quad
   S = \{ 0, 1, 2, {\lvert S \rvert} -1 \} \]

Mreža polja je lahko pravokotna, šestkotna ali celo kvazikristalna, tukaj se
bomo omejili na pravokotno mrežo. Na slošno je velikost polja lahko neskončna,
običajno pa se končna polja definira kakor pravokotnik velikosti \(N_x \times N_y\)
in skupno število celic v končnem polju je \(N=\lvert N_x \times N_y \rvert\).

Trenutno stanje neke celice \(c_{x,y}\) je odvisno od prejšnjega stanja njene okolice \(n_{x,y}\).
Podobno se bomo tudi pri obliki okolice omejili na pravokotnik velikosti \(M_x \times M_y\).
Število celic v okolici je \(m=\lvert M_x \times M_y \rvert\).
Na voljo je \({\lvert S \rvert}^m\) možnih okolic.

\[ n \in S^m
   \quad \textrm{and} \quad
   S^m = \{ 0, 1, 2, {\lvert S \rvert}^m -1 \} \]

Prostorski odnos med okolico in celico v prihodnjem stanju avtomata, ki jo ta okolica določa,
ni podrobno definiran. Običajno se smatra, da je celica v sredini okolice, ampak za opisani algoritem to ni
nujno pomembno.

\begin{figure}[htb]
\centerline{\psfig{figure=neighborhood,width=16cm}}
\caption[Velikost okolice.]{Okolica celice s podanima dimenzijama \(M_x=3\) in \(M_y=3\).}
\label{neighborhood}
\end{figure}

Preslikava sedanje okolice \(n_{x,y}^{t}\) v prihodnjo istoležno celico \(c_{x,y}^{t+1}\) je definiran
s tranzicijsko funkcijo \(f\).

\[ c_{x,y}^{t+1} = f(n_{x,y}^{t}) \]

Za potrebe iskanja predslik je zanimiva obratna funkcija \(f^{-1}\), ki ob
podanem stanju trenutne celice \(c_{x,y}^{t}\) vrne množico okolic,
ki se preslikajo v to vrednost.

\[ f^{-1}(c^{t}) = \{ n^{t-1} \in S^m \ \arrowvert \ f(n^{t-1}) = c^{t} \} \]

Dodaten pogoj za predsliko polja celic je, da se morajo okolice sosednjih celic
ujemati povsod, kjer se prekrivajo.

Podana konstrukcija mreže predslik in algoritem za izračun predslik omogočata uporabo
bolj splošne definicije, kjer je za vsako celico definiran lasten nabor predlik,
ki je neodvisen od stanja celice. Ta posplošitev je uporabljena za konstrukcijo umetnih
mrež predslik, ki poudarjajo konkretne probleme, ki definirajo kompleksnost algoritma.

\[ n^{t-1} \in S^m \]

Za dano polje celic velikosti \(N_x \times N_y\) in z odprtimi robovi,
je možno izračunati prihodnje stanje polja velikosti \((N_x-(M_x-1)) \times (N_y-(M_y-1))\).
V primeru, če so robovi polja ciklično zaprti (polje v obliki thorusa),
je pa polje prihodnjega stanja enako veliko kakor polje sedanjega.
Podobno velja za računanje predslik, za sedanje polje velikosti \(N_x \times N_y\)
je za odprte robove velikost polja predslik \((N_x+(M_x-1)) \times (N_y+(M_y-1))\).
Za ciklične robove pa sta velikosti enaki.

\chapter{Konstrukcija mreže predslik}

Mreža predslik je grafični konstrukt, ki omogoča upodobitev posameznih pojmov,
iz definicije celičnih avtomatov, kakor ločene grafične elemente. Odnosi med
temi elementi definirajo pravila na katerih se gradijo algoritmi za iskanje
predslik.

\section{De Bruijnov diagram}

Osnovani element grafične upodobitve je De Bruijinov diagram. V osnovi ta obravnava
ciklične premike končnih zaporedij simbolov, ter njihovo prekrivanje. Vozlišča v
diagramu so vsa možna končna zaporedja, povezave med njimi pa definirajo kako se
ta zaporedja prekrivajo med seboj.

Pri 1D celičnih avtomatih se problem neposredno preslika na De Bruijinov graf. McIntosh
in njegova skupina uporabljajo za analizo te De Bruijinove grafe neposredno. Sam sem pa
razvil modificiran graf, kjer so vozlišča podvojena, in gredo poti vedno od originala
proti dvojniku. To omogoča veriženje grafov, in razširitev osnovnega De Bruijinovega
grafa, ki opisuje okolico ene celice v verižen graf, ki opisuje verigo celic.

??? \cite{} uporabi podoben pristop za analizo 2D problema, tukaj pa je uporabljen nekoliko
drugačen pristop. Za potrebe opisa 2D celičnih avtomatov, je bila elementom
De Bruijinovega grafa dodana nova dimenzija. Povezave med vozliči se spremenijo
v ploskve, in vozlišča se spremenijo v robove ploskev.

Elementi celičnaga avtomata, ki se preslikajo v graf so:
\begin{itemize}[noitemsep,nolistsep]
\item nabor vseh možnih okolic celice postane nabor vseh ploskev (slika \ref{neighborhood_surfaces})
\item nabor prekrivanj okolic v smeri 2D dimenzij (slika \ref{overlap_dimension}) postane nabor povezav
\item nabor prekrivanj okolic v diagonalni smeri (slika \ref{overlap_diagonal}) postane nabor vozlišč
\end{itemize}

Elemente celičnega avtomata, kakor okolice in prekrivanja okolic je potrebno indeksirati,
tako da se lahko vsakemu elementu pripiše unikatna števna vrednost.
Vsaki okolici je pripisana zaporedna vrednost, ki je konstruirana kakor \(m\) mestno število
v \(S\)-iškem številskem sistemu (za podane primere dvojiški).
Cifre si sledijo od spodaj levo do zgoraj desno znotraj okolice.

\[ n = \sum |S|^{y N_x + x} \cdot c_{x,y} \]
\[ n \in \{0, 1, \dots N_x N_y -1\} \]

\begin{figure}[htb]
\centerline{\psfig{figure=neighborhood_index,width=8cm}}
\caption[Indeksiranje okolice.]{Indeksiranje okolice celice s podanima dimenzijama \(M_x=3\) in \(M_y=3\).}
\label{neighborhood_index}
\end{figure}

Celice se v smeri dimenzije X prekrivajo za ploskev velikosti \(O_x=M_x-1\) in \(O_y=M_y\) (slika \ref{overlap_dimension}).
To ob indeksiranju da nabor:
\[ o_x = \sum |S|^{y (M_x-1) + x} \cdot c_{x,y} \]
\[ o_x \in \{0, 1, \dots (M_x-1) N_y -1\} \]

Celice se v smeri dimenzije Y prekrivajo za ploskev velikosti \(O_x=M_x\) in \(O_y=M_y-1\) (slika \ref{overlap_dimension}).
To ob indeksiranju da nabor:
\[ o_y = \sum |S|^{y M_x + x} \cdot c_{x,y} \]
\[ o_y \in \{0, 1, \dots M_x (N_y-1) -1\} \]

\begin{figure}[htb]
\centerline{\psfig{figure=overlap_dimension,width=10cm}}
\caption[Prekrivaje okolic.]{Prekrivanje okolic sosednjih celic v smeri dimenzij X in Y, za velikost okolice \(M_x=M_y=3\).}
\label{overlap_dimension}
\end{figure}

Celice se v diagonalni smeri prekrivajo za ploskev velikosti \(O_x=M_x-1\) in \(O_y=M_y-1\) (slika \ref{overlap_diagonal}).
To ob indeksiranju da nabor:
\[ o_{xy} = \sum |S|^{y (M_x-1) + x} \cdot c_{x,y} \]
\[ o_{xy} \in \{0, 1, \dots (M_x-1) (N_y-1) -1\} \]

\begin{figure}[htb]
\centerline{\psfig{figure=overlap_diagonal,width=10cm}}
\caption[Prekrivanje okolic - diagonalno.]{Prekrivanje okolic sosednjih celic v diagonalni smeri, za velikost okolice \(M_x=M_y=3\).}
\label{overlap_diagonal}
\end{figure}

Nastali graf (slika \ref{network_single}) ima poleg vozlišč in povezav med njimi tudi ploskve. Ploskve bi v teoriji grafov
opisali kakor zanke v grafu, z dodano omejitvijo, da mora vsako vozlišče in povezava v zanki
pripadati drugemu prekrivanju okolic.

\begin{figure}[htb]
\centerline{\psfig{figure=network_single,width=5cm}}
\caption[Mreža ene celice.]{Mreža ene celice za binarni CA z okolico quad \(M_x=M_y=2\).}
\label{network_single}
\end{figure}

\begin{figure}[htb]
\centerline{\psfig{figure=neighborhood_surfaces,width=16cm}}
\caption[Nabor ploskev.]{Nabor ploskev za vse možne okolice za binarni CA z okolico quad \(M_x=M_y=2\).}
\label{neighborhood_surfaces}
\end{figure}

\section{Mreža}

Diagrame, ki opisujejo preteklost posamezne celice, je možno sestaviti v mrežo,
ki opisuje polje več celic. Nabor predslik celotnega polja je ekvivalenten naboru
vseh zveznih ploskev, ki prekrivajo celotno polje in, ki jih je možno sestaviti iz
naborov ploskev diagramov posameznih celic.

Preslikava iz zvezne ploskve v mreži predslik v konfiguracijo polja celic predslike je enolična.
Najlažje je razumeti preslikavo za okolico velikosti \(3 \times 3\),
od vsakega odseka ploskve za posamezno celico se vzame centralno celico,
na koncu pa se doda še za eno celico širok rob okoli celotne ploskve.

Podani primeri uporabljajo binarni CA z quad okolico.
Za ta CA je velikost diagonalnega prekrivanja okolic ena sama celica,
Posledično ima nabor vozlišč le dve vrednosti, ki neposredno predstavljajo
vrednosti celic v predsliki (slika \ref{network_array}).

Razširitev diagrama ene celice v mrežo lahko dokažemo z indukcijo.
Dokazati želimo, da je neka konfiguracija celic predslika dane sedanjosti
če in samo če je ta ekvivalentna zvezni ploskvi v mreži predslik.

\textbf{Prvi element:}
Iz definicije velja, da je za eno samo celico v mreži nabor ploskev enak naboru vseh predslik.
\textbf{Naslednji element:}
Obstoječi mreži predslik za več celic z danim naborom zveznih ploskev dodamo novo celico.
Ploskev iz nabora dodane celice se zvezno veže z obstoječim naborom zveznih ploskev natanko v primeru,
ko se z eno od ploskev ujema v robu (povezavi med vozliščema).
Temu je tako, ker so indeksi vozlišč in povezav ekvivalentni vrednosti prekrivanj okolic.

\begin{figure}[htb]
\centerline{\psfig{figure=network_array,width=14cm}}
\caption[Mreža polja celic.]{Mreža velikosti \(N_x=3\) in \(N_y=3\) za binarni CA z okolico quad.
Poudarjena je ena zvezna ploskev (konfiguracija pripadajoče predslike je izpisana) in njen rob.}
\label{network_array}
\end{figure}

\section{Robni pogoji} 

Pri 1D CA so robni pogoji definirani na dveh koncih,
ki omejujejo končno število celic na neskončni premici.
Če je 1D CA deginiran kakor daljica je robni pogoj samo eden.
Robni pogoj definira katere okolice (vozlišča pri mreži predslik za 1D CA)
in s kakšnimi utežmi so na voljo ob robu.
Lahko si jih predstavljamo tudi kakor vpliv, ki ga ima neskončen poltrak celic,
ki sega izven roba opazovane konfiguracije.

Pri 2D CA je robni pogoj definiran na sklenjeni poti okoli ploskve opazovane konfiguracije celic.
V mreži predslik za 2D CA povezave med vozlišči definirajo rob ploskve (slika \ref{network_array}).
Na splono ima vsaka ploskev v mreži svoj rob, robni pogoj določa kako je ta ploskev obravnavana.
Ker uporabna vrednost splošnega robnega pogoja še ni znana,
in bi splošnost izrazito povečala zahtevnost algoritma za iskanje predslik,
so tukaj vsi robovi obravnavani enako. Temu bomo rekli odprt rob,
ker ta ne definira nobenih omejitev, katere zvezne ploskve v mreži predslik
so dovoljene in katere ne.

Obstaja še eden enostaven robni pogoj, ki je definiran za ciklično sklenjena končna CA polja.
Ta tip robnega pogoja tukaj ne bo obravnavan, ker še dodatno poveča kompleksnost algoritmov.

\chapter{Algiritem za štetje in izpis predslik}

\section{Robni pogoji}

\section{Procesiranje v eni dimenziji}

\section{Procesiranje v drugi dimenziji}

\section{Izpis predslik}


\chapter{Primerjava z znanimi algoritmi}

\section{Procesna zahtevnost}


\subsection{Sklepne ugotovitve}

Sklepne ugotovitve naj prikažejo oceno o opravljenem delu in povzamejo težave, na katere je naletel kandidat. Kot rezultat dela
lahko navede ideje, ki so nastale med delom, in bi lahko bile predmet novih raziskav.

\newpage

%********************************************

\appendix

%\addcontentsline{toc}{chapter}{\protect Dodatki}

\newpage

\addcontentsline{toc}{chapter}{Seznam slik}
\addtocontents{toc}{\protect\vspace{-2ex}}
\listoffigures

\newpage

\addcontentsline{toc}{chapter}{Seznam tabel}
\listoftables

%\listofalgorithms

%********************************************

\newpage

\bibliographystyle{slplainurl}
\addcontentsline{toc}{chapter}{Literatura}
\label{stran_literatura}
\bibliography{magisterij} 

\end{document}




