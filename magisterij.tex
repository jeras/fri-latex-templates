%*****************************************************************
%   Vzorec za pisanje diplomskega dela,
%   ki vsebuje navodila za izdelavo diplomskega dela
%
%   UNIVERZA V LJUBLJANI
%   Fakulteta za računalništvo in informatiko
%
%   Pripravila: Peter.Peer@fri.uni-lj.si
%               Franc.Solina@fri.uni-lj.si
%*****************************************************************

\documentclass[12pt,a4paper,openany]{book}

%Uporabljeni paketi
\usepackage[utf8]{inputenc}
\usepackage{cmap}
\usepackage{type1ec}
\usepackage[T1]{fontenc}
\usepackage{fancyhdr}
\usepackage{graphicx,epsfig}
\usepackage[slovene]{babel}
\usepackage{cite}

\usepackage[pdftex,colorlinks,citecolor=black,filecolor=black,linkcolor=black,urlcolor=black,pagebackref]{hyperref}
\usepackage{tikz}

%Velikost strani - dvostransko
\oddsidemargin 1.4cm
\evensidemargin 0.35cm
\textwidth 14cm
\topmargin 0.26cm
\headheight 0.6cm
\headsep 1.5cm
\textheight 20cm

%Nastavitev glave in repa strani
\pagestyle{fancy}
\fancyhead{}
\renewcommand{\chaptermark}[1]{\markboth{\textsf{Poglavje \thechapter:\ #1}}{}}
\renewcommand{\sectionmark}[1]{\markright{\textsf{\thesection\  #1}}{}}
\fancyhead[RE]{\leftmark}
\fancyhead[LO]{\rightmark}
\fancyhead[LE,RO]{\thepage}
\fancyfoot{}
\renewcommand{\headrulewidth}{0.0pt}
\renewcommand{\footrulewidth}{0.0pt}

\newcommand{\gnuplot}{\textbf{gnuplot}}
\newcommand{\pgfname}{\textsc{pgf}}
\newcommand{\tikzname}{Ti\emph{k}Z}

\input{cc}

%********************************************

\begin{document}

% stran 1 med uvodnimi listi
\thispagestyle{empty} 

\begin{center}
{\large 
UNIVERZA V LJUBLJANI\\
FAKULTETA ZA RAČUNALNIŠTVO IN INFORMATIKO\\
}

\vspace{3cm}
{\LARGE Iztok Jeras}\\

\vspace{2cm}
\textsc{\textbf{\LARGE 
Predslike 2D celičnih avtomatov
}}

\vspace{2cm}
{ MAGISTERSKO DELO}\\
{ NA UNIVERZITETNEM ŠTUDIJU}\\

\vspace{2cm} 
{\Large Mentor: prof. dr. Branko Šter}

\vfill
{\Large Ljubljana, 2016}
\end{center}

\newpage

\ \thispagestyle{empty}

\newpage

%********************************************

% stran 2 med uvodnimi listi
\thispagestyle{empty}

\vspace*{5cm}
{\small \noindent
To magistrsko delo je ponujeno pod licenco \textit{Creative Commons Priznanje avtorstva-Deljenje pod enakimi pogoji 2.5 Slovenija}
ali (po želji) novejšo različico.
To pomeni, da se tako besedilo, slike, grafi in druge sestavine dela kot tudi rezultati diplomskega dela lahko prosto distribuirajo,
reproducirajo, uporabljajo, dajejo v najem, priobčujejo javnosti in predelujejo, pod pogojem, da se jasno in vidno navede avtorja in naslov tega
dela in da se v primeru spremembe, preoblikovanja ali uporabe tega dela v svojem delu, lahko distribuira predelava le pod
licenco, ki je enaka tej.
Podrobnosti licence so dostopne na spletni strani \url{http://creativecommons.si/} ali na Inštitutu za
intelektualno lastnino, Streliška 1, 1000 Ljubljana.

\begin{center}% 0.66 / 0.89 = 0.741573033707865
  \CcImageCc{0.741573033707865}\hspace*{1ex}\CcGroupBySa{1}{1ex}
\end{center}
}

\vspace*{1.5cm}
{\small \noindent
Izvorna koda diplomskega dela, njenih rezultatov in v ta namen razvite programske opreme je ponujena pod GNU General Public License,
različica 3 ali (po želji) novejšo različico. To pomeni, da se lahko prosto uporablja, distribuira in/ali predeluje pod njenimi pogoji.
Podrobnosti licence so dostopne na spletni strani \url{http://www.gnu.org/licenses/}.
}

\begin{center} 
\ \\ \vfill
{\em
Besedilo je oblikovano z urejevalnikom besedil \LaTeX. \\ Slike so izdelane s pomočjo jezika \pgfname/\tikzname. \\ Grafi so narisani
s pomočjo programa \gnuplot.}
\end{center}

\newpage

\ \thispagestyle{empty}

\newpage

%********************************************

% stran 3 med uvodnimi listi
\thispagestyle{empty}

Namesto te strani {\bf vstavite} original izdane teme diplomskega dela s podpisom mentorja in dekana ter \v zigom fakultete, ki ga diplomant
dvigne v študent\-skem referatu,  preden odda izdelek v vezavo!

\newpage

%********************************************

% stran 4 med uvodnimi listi je prazna 
\ \thispagestyle{empty}

\newpage

%********************************************

% stran 5 med uvodnimi listi

\thispagestyle{empty}

\vspace{1cm}
\begin{center} 
{\Large \textbf{IZJAVA O AVTORSTVU}}
\end{center}

\begin{center} 
{\Large diplomskega dela}
\end{center}

\vspace{1cm}
Spodaj podpisani \hspace{0.5cm} Iztok Jeras,

\vspace{0.5cm}
z vpisno številko \hspace{0.5cm} 63030393,

\vspace{1cm}
sem avtor diplomskega dela z naslovom:
   
\vspace{0.5cm}
Predslike 2D celičnih avtomatov

\vspace{1.5cm}
S svojim podpisom zagotavljam, da:
\begin{itemize}
	\item sem diplomsko delo izdelal samostojno pod mentorstvom 
	
	prof. dr. Branko Šter
	
	in somentorstvom 
	
	prof. [doc.] dr. Ime Priimek
	
	\item so elektronska oblika diplomskega dela, naslov (slov., angl.), povzetek (slov., angl.) ter ključne besede (slov., angl.) identični s tiskano obliko diplomskega dela
	\item soglašam z javno objavo elektronske oblike diplomskega dela v zbirki ''Dela FRI''.
\end{itemize}

\vspace{1cm}
V Ljubljani, dne xx.xx.2016 \hspace{1cm} Podpis avtorja/-ice:

\newpage 

%********************************************

% stran 6 med uvodnimi listi je prazna pri dvostranskem tiskanju
\ \thispagestyle{empty}

\newpage

%********************************************

% stran 7 med uvodnimi listi

\chapter*{Zahvala}

\thispagestyle{empty}

Na tem mestu se diplomant zahvali vsem, ki so kakorkoli pripomogli k uspešni izvedbi diplomskega dela.


\newpage

%********************************************

% stran 8 med uvodnimi listi je prazna pri dvostranskem tiskanju
\ \thispagestyle{empty}

\newpage

%********************************************

% stran 9 med uvodnimi listi
\thispagestyle{empty}

$\;$ 

\vspace{5cm}
\hfill {\Large \em Morebitno posvetilo}
\thispagestyle{empty}

\newpage

%********************************************

% stran 10 med uvodnimi listi je prazna pri dvostranskem tiskanju

\ \thispagestyle{empty}

\newpage

%********************************************

\renewcommand\thepage{} 
\tableofcontents 
\renewcommand\thepage{\arabic{page}}

\thispagestyle{empty}


%********************************************

\chapter*{Seznam uporabljenih kratic in simbolov}

\thispagestyle{empty}

2D dvo dimenzionalni
CA celični avtomati
GoL Game of Life (igra življenja)
GoE Garden of Eden (stanje brez predslik)

%\cleardoublepage

\clearpage{\pagestyle{empty}\cleardoublepage}

%********************************************
%zacno se glavni listi, ki so numerirani z arabskimi stevilkami

\setcounter{page}{1}
\pagenumbering{arabic}

\chapter*{Povzetek}

\addcontentsline{toc}{chapter}{Povzetek}

Medtem ko je računanje predslik 1D celičnih avtomatov dobro raziskan in
podrobno dokumentiran problem, je to področje pri 2D celičnih avtomatih manj
raziskano. To magistersko delo poizkuša aplicirati metode razvite za 1D avtomate
na 2D problem. Prikazan je algoritem, ki omogočajo štetje in izpis predslik.
Razvit je bil s pomočjo grafičnega modela, ki omogoča uporabo teorije grafov pri izračunih in dokazih.

\vspace{1.3cm}
\noindent
{\large \bf Ključne besede:}

\vspace{0.5cm}
\noindent
celični avtomati, predslike, procesna zahtevnost, reverzibilnost, trid, quad


\chapter*{Abstract}

\addcontentsline{toc}{chapter}{Abstract}

While computing preimages of 1D cellular automata is a well researched and
documented problem, for 2D cellular automata there is less research available.
This masters thesis attempts to apply methotds developed for 1D automata to the
2D problem. An algorithm is shown, which can count and list preimages.
It was developed with the help of a graphical representation, which enables using graph theory for computation and proofs.

\vspace{1.3cm}
\noindent
{\large \bf Key words:}

\vspace{0.5cm}
\noindent
cellular automata, preimages, computational complexity, reversibility, Garden of Eden, Game of Life, trid, quad


%********************************************

\chapter{Uvod}

\section{Celični avtomat kakor model vesolja}

Ker lahko vsak univerzalen sistem modelira vsak drugi univerzalen sistem, predpostavimo,
da lahko z univerzalnimi CA modeliramo vesolje. Samo modeliranje vesolja
je še izven našega dosega, poizkuša pa se vsaj doseči povezave z teoretično fiziko.
S stališča informacijske teorije in termodinamike je predvsem zanimiv model gravitacije
kakor entropijske sile (Entropic gravity \cite{Verlinde2010}), ki predpostavlja, da je
3D vesolje projekcija procesov, ki se odvijajo na 2D ploskvi. Podobno CA omogočajo
opazovanje abstraktnega kopiranja informacij (replikacija) in evolucijo \cite{Salzberg2004}.

\section{Informacijska dinamika}

Informacijsko dinamiko CA se najpogosteje opisuje samo kakor reverzibilno ali ireverzibilno,
obstaja tudi nekaj člankov, ki opazujejo entropijo sistema.
Pogosto je tudi opazovanje dinamike delcev pri Game of Life ali elementarnem pravilu 110.
Ne obstaja pa še splošna teorija dinamike informacij v CA.
V svojem članku \cite{JerasDobnikar2007} in prispevkinh na konferencah \cite{DBLP:conf/iccS/JerasD06, DBLP:conf/automata/Jeras08},
sem grafično upodobil predslike trenutnega stanja za 1D problem.
Iz upodobitve je videti, da se ponekod izgubi več informacije kakor drugod,
kar kaže na možnost izpeljave kvalitativne in kvantitativne teorije dinamike informacij;
žal se ta možnost še ni udejanila. Podobno je možno grafično upodobiti predslike 2D CA,
ter iz grafov sklepati o izgubi informacij v 2D CA.

\section{Problem predslik 2D CA}

Najbolje teoretično raziskan 2D CA je GoL (Game of Life ali slovensko igra živjenja).
Ogromno truda je bilo vloženega v raziskovanje delcev in njihove dinamike. S pomočjo
osnovnih gradnikov, je mogoče skonstruirati kompleksnejše sisteme, med katerimi so
najzanimivejši turingov stroj /cite{} in univerzalni konstruktor /cite{}.

Delci so dejansko atraktorji v razvoju CA na končni običajno periodični mreži (thorus).
Algoritmi za iskanje predslik so uporabljeni za določitev atraktorjevega korita /cite{}.
V ireverzibilnem celičnem avtomatu se pojavljajo stanja brez predslik imenovana GoE
(Garden of Eden ali rajski vrt). Pomensko so GoE nasprotje delcev, saj se nahajajo kar
najdlje od atraktorja na robu korita. GoE stanja prav tako kakor delci privlačijo raziskovalce /cite{},
čeprav v manši meri kakor delci.

Največ raziskav s področja predslik GoL je bilo opravljenih ravno s ciljem iskanja GoE stanj /cite{}.
S stališča algoritma za štetje predlik je GoE stanje tako, ki ima 0 preslik.
Algoritem za štetje predslik je možno pretvoriti v algoritem za preverjanje ali je stanje GoE,
tako da se opracije nad celimi števili pretvori v logične opreracije nad Boolovimi staji.

Raziskave algoritmov sem se lotil s predpostavko, da je možno opraviti štetje s zahtevnostjo,
ki je linearno odvisna od velikosti problema \( O(n) \)  (n je število opazovanih celic). Čeprav je to možno pri 1D problemu,
se izkaže da 2D problem ni tako preprost. Predstavljeni so primeri iz katerih je razvidno,
da algoritem z linearno zahtevnostjo ne more pravilno opisati vseh situacij.
Zahtevnost opisanega algotitma sicer raste s kvadratom eksponenta eksponentno z eno
od dimenzij CA mreže  \( O(2^2n) \), ja pa možno da obstaja algiritem z nižjo kompleksnostjo.

Opisan algoritem je primerjan z ostalimi doslej znanimi algoritmi. Čeprav s stališča procesne
zahtevnosti ne prinaša želenega napredka, pa to da temelji na pregledni grafični upodobitvi
daje upanje, da bojo razne optimizacije razvidne bodočim raziskovalcem.

Paulina Léon in Genaro Martínez \cite{PaulinaGenaro2016} poizkušata aplicirati
De Bruijn-ove diagrame na 2D celične avtomate, doslej je bilo to orodje uporabljeno le na 1D problemih.
Točneje, opazujeta dva celična avtomata:'Game of Life' in 'Diffusion rule',
s poudarkom na opazovanju stabilnih delcev. De Bruijn-ovi grafi so tudi osnova mojih raziskav,
so pa drugače grafično upodobljeni, tako da se jih lahko poveže v opis celotnega celičnega sistema,
in niso omejeni na opis predstanj ene same celice.

Razni avtorji \cite{Hartman2013} iščejo vzorce tipa 'Garden of Eden' v celičnem avtomatu 'Game of Life'.
Zanimiv je pristop s teorijo končnih avtomatov in regularnih jezikov, ki je v
osnovi namenjen eno dimenzionalnim sistemom. Jean Hardouin-Duparc ga je razširil
tako, da je celice iz vrstice 2D polja združil v simbole regularnega jezika, zaporedje več
vrstic pa predstavlja besedo. Originalni članek je v francoščini, zato še iščem članek,
kjer bi pristop opisal v angleščini. Podoben pristop s končnimi avtomati nameravam uporabiti tudi sam.



Doslej sem že razvil napredne algoritme za izračun predslik 1D sistema.
Skozi zgodovino so taki algoritmi napredovali, tako da je padala njihova
procesna zahtevnost in opisna/implementacijska zahtevnost.
\begin{enumerate}[noitemsep,nolistsep]
\item 'brute force' algoritmi \( O(c^n) \)
\item improvizirani algoritmi
\item zasnove matematičnega modela
\item optimalni algoritmi \( O(n \log(n)) \) ali celo \( O(n) \)
\end{enumerate}
Iskanje slik 2D sistema je trenutno nekje med improvizacijo in matematičnim modelom.
Z magistrsko nalogo bi rad razvil algoritme, ki se nagibajo k optimalnosti.

Magistrsko delo bo obsegalo matematičen model, ki bo predvidoma temeljil na matričnih operacijah,
kjer matrike predstavljajo grafe in končne avtomate.
Za lažje razumevanje bo problem tudi grafično predstavljen.
Algoritem bo implementiran kot računalniški program,
ki bo omogočal tudi izris grafične predstavitve problema.

Primerjava s sorodnimi deli bo s stališča procesne zahtevnosi algoritma in glede na to,
katere znane probleme bo algoritem sposoben rešiti. Nekaj takih problemov, urejenih glede na zahtevnost, je:
\begin{enumerate}[noitemsep,nolistsep]
\item določitev, ali obstajajo predslike za dano trenutno stanje sistema
\item štetje predslik
\item naštevanje konfiguracij predslik
\item jezik vseh stanj brez predslik
\item vprašanje reverzibilnosti sistema
\end{enumerate}

Rešitev problema določitve obstoja predslik si že predstavljam. 
Predvidevam, da bom uspel rešiti še problem preštevanja predslik,
in ker je to manjši korak, tudi njihovo naštevanje.

Preostalih problemov se tokrat ne bom loteval.
Problem jezika stanj brez predslik bi potreboval teorijo 2D formalnega jezika, ki še ne obstaja.
Poleg tega je povezan s problemom reverzibilnosti, ki je na spološno dokazano nerešljiv \cite{Kari1989}.


\chapter{Konstrukcija mreže predslik}

\section{De Bruijnov diagram}

\section{Mreža}

\section{Robni pogoji} 


\chapter{Algiritem za štetje in izpis predslik}

\section{Robni pogoji}

\section{procesiranje v eni dimenziji}

\section{Procesiranje v drugi dimenziji}

\section{Izpis predslik}


\chapter{Primerjava z znanimi algoritmi}

\section{Razgledanost}

\section{Jezik diplomskega dela}


\subsection{Sklepne ugotovitve}

Sklepne ugotovitve naj prikažejo oceno o opravljenem delu in povzamejo težave, na katere je naletel kandidat. Kot rezultat dela
lahko navede ideje, ki so nastale med delom, in bi lahko bile predmet novih raziskav.

\subsection{Priloge}
Priloge (slike, diagrami, algoritmi, načrti), 
če so potrebne, kandidat izdela kot posebna poglavja (dodatki), ki jih zaradi preglednosti ni smiselno vključiti v glavni del naloge. Vsi
dodatki morajo biti naslovljeni in oštevilčeni, običajno z velikimi tiskanimi črkami. 

Na tem mestu je priporočljivo navesti tudi seznama slik in tabel.

\subsection{Viri}
Vire navede kandidat po abecednem vrstnem redu priimkov avtorjev. Navajajo naj se le tisti viri, na katere se
kandidat v besedilu sklicuje! Dolg seznam ni dokaz, da ima kandidat tudi tak pregled čez literaturo. Za navajanje virov naj se uporabi
slovenska inačica formata IEEE. V tem formatu se viri (članek v zborniku konference,  knjiga, članek v reviji in spletna stran) navedejo  kot
je to prikazano v poglavju Literatura (stran \pageref{stran_literatura}). Sklicevanje na vir se v besedilu naloge označi z zaporedno številko
vira v oglatem oklepaju, npr. \cite{peytonjones93} oziroma \cite{trucco98,wadler89,IEEE}, kadar se na istem mestu sklicuje na več
različnih virov.  Zglede za druge vrste virov je mogoče dobiti v revijah IEEE, ki so na voljo v knjižnici, ali na spletnem naslovu
\cite{IEEE}.

\newpage

%********************************************

\appendix

%\addcontentsline{toc}{chapter}{\protect Dodatki}

\chapter{Kaj so priloge ali dodatki}

Priloge (slike, diagrami, algoritmi, načrti), 
če so potrebne, kandidat izdela kot posebna poglavja (Dodatek A, Dodatek B, \ldots), ki jih zaradi preglednosti ni smiselno vključiti v glavni
del naloge. Vsi dodatki morajo biti naslovljeni in oštevilčeni, običajno z velikimi tiskanimi črkami. 

\newpage

\addcontentsline{toc}{chapter}{Seznam slik}
\addtocontents{toc}{\protect\vspace{-2ex}}
\listoffigures

\newpage

\addcontentsline{toc}{chapter}{Seznam tabel}
\listoftables

%\listofalgorithms


%********************************************

\newpage

\bibliographystyle{slplainurl}
\addcontentsline{toc}{chapter}{Literatura}
\label{stran_literatura}
\bibliography{magisterij} 


\end{document}




